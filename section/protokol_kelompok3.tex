%Resume protokol
%kelompok 4 D4 TI-2D 
%Ayu Permata Sari        1154022 
%Librantara Erlangga     1154071 
%Martin Luter Zega       1154120 
%Putri Aulia Ramadhanie  1154096 
%Ryan Hafizh Herdiana    1154067 


\section{Pengenalan Protokol} 
  Penanggalan merupakan salah satu sebuah mahakarya yang bisa ditemukan oleh umat manusia. Manusia mempelajari dan memanfaatkan alam [Matahari,Bulan dan Bintang] untuk menghitung pergantian tanggal,bulan dan juga tahun. 
umumnya penanggalan digunakan untuk mengetahui waktu yang telah dilewati oleh umat manusia. Adanya sistem penanggalan ini membuat manusia dapat mengingat seluruh kejadian dan pristiwa yang terjadi di dunia ini. 
Menurut artikel dari setyanto berdasarkan benda langit yang digunakan sebagai dasar perhitungan sistem penanggalan dapat dikategorikan menjadi 3 kelompok yaitu:\cite{setyanto2015kriteria} 

  \subsection{} 
   
    
  \subsection{Solar calendar/Kalender Surya} 
    Kalender surya menggunakan pergerakan bumi mengelilingi matahari sebagai acuannya.Sistem kalender surya ini biasa digunakan oleh orang-orang eropa. Beberapa contoh kalender yang menggunakan sistem in
